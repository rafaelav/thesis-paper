\chapter{Discussion and future work}

During the work conducted for our study we have come across a number of
interesting findings and observations. This chapter contains a summation of
these surfacing results as well as a list of possible directions and subtopics
that can be explored during future research on the subject.

In order to capture results that can reflect general facts about the way human
mobility patterns emerge, it is important for the conducted studies to
incorporate the usage of a high amount of data. Despite the quantity of data
being relevant for the results, the importance of the quality of the data can
exceed it. Considering this, a thorough cleaning of the Wifi data proved to be
of a high importance for our present research. By eliminating the noise and the
excessive signals from various sources that were not of interest for the present
work, we have obtained data that allowed us to make accurate observations on
human mobility.

Despite the data clean up, it has been easy to notice from the very beginning
that the signal strength received from the high number of access points does not
tend to stay constant even if the location seems to be maintained by the user.
Interferences influence the signal strength \cite{MahantiCWA10} and can
complicate the algorithm used for identifying locations based both on access
points BSSIDs and RSS values. We have analyzed the possibility of using
different average signal calculations (Sections \ref{average_sig} and
\ref{running_avg}) in order to smooth out the spikes caused by interferences.
The averaging, especially the running average, did improve the data by removing
a part of the spikes, however this proved to not be sufficient for the present
case. Previous research projects show that an approach that does not necessarily
require the use of signal strength for the visible access points, but just the
knowledge of the signals being present or not in a given point can be a good
starting point for determining a location
\cite{Larsen:2009:MCT:1813042.1813063}. We have, therefor, determined the
presence of the various access points across different time frames and we have
used the results in order to explore the possibility of determining locations
based on this information.

It is worth mentioning that additional work can be done for improving our
understanding of interferences and noise that can cause problems when working
with Wifi data. The present paper presented a few techniques which have been used in
order to eliminate possible noise, however, future improvements can be added for
obtaining data which has an even higher qualitative value.

In order to extract locations from Wifi data we have worked with three different
algorithms: based on networks, k-means clustering and based on the use of Hidden
Markov Models. The algorithm that uses the construction of a network based on
the presence of the access points at given times did not lead to acceptable
results for the duration of our study. However, the k-means clustering and the
algorithm based on Hidden Markov Models combined with a k-fold cross validation
algorithm in order to estimate the correct number of locations that were to be
expected for a user lead to good approximations for stop locations. For the data
we have used, the algorithm based on Hidden Markov Models, has statistically
given better approximations, but further research is needed in order to
determine if either of the two methods is better than the other or they just
behave better in various circumstances. Another thing which is worth taking into
consideration for further research is that, even though during our study, we did
not reach acceptable results by the use of the network method, further research
is needed in order to determine if the idea behind this approach can produce
note worthy findings. The benefits are that, by identifying clear
characteristics of the locations, this enables the construction of an algorithm
that can have a lower execution time for extracting locations from large amounts
of data. Also, an additional interesting focus point can be represented by
researching the implementation of an improved method that can be used for
estimating the expected number of locations for a given time frame. The present
paper has presented the use of cross validation as a method of determining the
best possible estimation, however, Hidden Markov Models constitute a highly
effective model which, possibly, with further research can be able to eliminate
the use of estimation and allow the exact determination of the number of
locations which can be observed during a given sequence of data. Other options
can be explored and tested, as well, for achieving the best results.

Due to the using of cross validation in order to determine the approximate
number of locations we are to expect for each user during a given time frame, as well
as the complexity of the location determination algorithms, the execution time
rises exponentially if the amount of input data is high. In order to avoid this
problem we have determined locations over short one day intervals. The locations
estimated for different days needed to be compared and analyzed in order to
determine when they coincided. This offered us the opportunity to explore
different methods in order to determine the location matching (Chapter
\ref{ch_matching}). We have also compared our extracted locations with locations
extracted from GPS data and the results support that, even though the GPS data
offers better location estimations, the Wifi data can provide acceptable results
as well. Even though we have taken into consideration various possibilities that
can be used in order to determine if two locations which have been identified in
different iterations by a location extraction algorithm can, in fact, be
considered to be the same location, further work can be done on this subject.
Even though the estimations which the present solution gives are accurate most
of the times, additional improvements can be made and further research can lead
to the development of a new method that can exceed the benefits of the one that
has been used in this case.

The locations extracted from a selection of $65$ of users from our original pool
of $131$ users has allowed us to explore the predictability of human travel
trajectories. The results we have reached support previous studies
(\cite{Barabasi10}, \cite{Barabasi08}, \cite{Sinatra14}, \cite{Brockmann06}
etc.) which indicate that the mobility patterns we display seem to have a high
degree of predictability. This can prove that, in fact, we could be less
spontaneous than we believe we are and that our behaviour is deeply rooted in
habits. However, further research can still be done and is needed for estimating
the predictability of human mobility. The results we have obtained are based on
the data that has been collected through a period of one month and is provided
by a focus group of $65$ people who have in common the fact that they are
students at the Technical University of Denmark. Further research on a larger
group of people possibly with various backgrounds could give additional
interesting results and would be needed in order to make definitive affirmations
about emerging mobility and predictability patterns.

During our work we have debated what is a good approximation for the time an
user can be expected to stay at a given location in order to consider it a stop
location rather than a transit location. We have proposed the use of a $5$
minute time frame, however further studies can be conducted in order to estimate
what is the average time a person spends at a given location based on the
location type. This can lead to further considerations and adaptations for
algorithms that try to extract stop locations both from Wifi data, but also from
other types of data, like ,for example, GPS data.

Clearly the path of exploring the subject of human mobility and predictability
of human travel trajectories is far from reaching an end. Numerous studies, the
present one included, have been conducted on this topic, however there are
still an enormous number of questions that need answers and further work is
needed in order to be able to fully understand what determines our trajectories
and how we can use the knowledge in order to increase our quality of life. The
topic continues to remain of high interest as the possible results have an
applicability potential that expands over a variety of fields and domains such
as the prevention of the spreading of epidemics, the design of better urban and
transportation infrastructures, and many other areas of interest.
\chapter{Discussion}

During the work conducted for our study we have come across a number of
interesting findings and observations. This chapter contains a summation of
these surfacing results.

In order to capture results that can reflect general facts about the way human
mobility patterns emerge, it is important for the conducted studies to
incorporate the usage of a high amount of data. Despite the quantity of data
being relevant for the results, the importance of the quality of the data can
exceed it. Considering this, a through cleaning of the Wifi data proved to be of
a high importance for our present research. By eliminating the noise and the
excessive signals from various sources that were not of interest for the
present work, we have obtained data that allowed us to make interesting
observations on human mobility.

Despite the data clean up, it has been easy to notice from the very beginning
that the signal strength received from the high number of access points does not
tend to stay constant even if the location seems to be maintained by the user.
Interferences influence the signal strength \cite{MahantiCWA10} and can
complicate the algorithm used for identifying locations based both on access
points BSSIDs and RSS values. We have analyzed the possibility of using
different average signal calculations (Sections \ref{average_sig} and
\ref{running_avg}) in order to smooth out the spikes caused by interferences.
The averaging, especially the running average, did improve the data by removing
a part of the spikes, however this proved to be not sufficient for the present
case. Previous research projects show that an approach that does not necessarily
require the use of signal strength for the visible access points, but just the
knowledge of the signals being present or not in a given point can be a good
starting point for determining a location
\cite{Larsen:2009:MCT:1813042.1813063}. We have, therefor, determined the
presence of the various access points across different time frames and we have
used the results in order to explore the possibility of determining locations
based on this information.

We have worked with three different algorithms in order to extract locations
from Wifi data: based on networks, k-means clustering and based on the use of
Hidden Markov Models. The algorithm that uses the construction of a network
based on the presence of the access points at given times did not lead to
acceptable results for the duration of our study. However, the k-means
clustering and the algorithm based on Hidden Markov Models combined with a
k-fold cross validation algorithm in order to estimate the correct number of
locations that were to be expected for a user lead to good approximations for
stop locations. For the data we have used, the algorithm based on Hidden Markov
Models, has statistically given better approximations, however further research
is needed in order to determine if either of the two methods is better than the
other or they just behave better in various circumstances.

Due to the need of using cross validation in order to determine the approximate
number of locations we are to expect for each user during a given time frame, as
well as the complexity of the location determination algorithms, the execution
time rises exponentially if the amount of input data is high. In order to avoid
this problem we have determined locations over short one day intervals. The
locations estimated for different days needed to be compared and analyzed in
order to determine when they coincided. This offered us the opportunity to
explore different methods in order to determine the location matching (Chapter
\ref{ch_matching}). We have also compared our extracted locations with locations
extracted from GPS data and the results support that, even though the GPS data
offers better location estimations, the Wifi data can provide acceptable
results as well.

The locations extracted from a selection of $65$ of user from our original pool
of $131$ users has allowed us to explore the predictability of human travel
trajectories. The results we have reached support previous studies
(\cite{Barabasi10}, \cite{Barabasi08}, \cite{Sinatra14}, \cite{Brockmann06}
etc.) which indicate that the mobility patterns we display seem to have a high
degree of predictability. This can prove that, in fact, we could be less
spontaneous than we believe we are and that our behaviour is deeply based on
habits.

The present work proves that the exploration of human mobility is currently just
at the beginning. Despite the numerous previous projects that have approached
the subject, there are still an enormous number of questions that need answers
and further work is needed in order to be able to fully understand what
determines our trajectories and how can we use the knowledge in order to
increase our quality of life either by using it to prevent the spreading of
diseases, to design better urban and transportation or any other way that can
prove to help us.

\chapter{Locations}
\label{locations}
% 20 pages
Human mobility has been attracting a high degree of attention from numerous
study fields among which we find urban and traffic planning, traffic prediction,
the spreading of diseases and many others \cite{AsgariGB13} \cite{Brockmann08}.

The studies that have been conducted on this subject have been using various
ways to identify the travel behaviour of people. Some of them have focused on
studying the information gathered from observing the way in which money is
dispersed through time \cite{Brockmann06}, or they have been focusing in
studying the behaviour of mobile phone users by analyzing the way they move
based on the communication towers their phones are connecting to when they are
engaging in voice communication \cite{Barabasi08}. There are studies that try to
understand human mobility through the glass of social networks
\cite{yang2010using}, as it can be observed that individuals prefer to meet with
other people that are part of their community more often
\cite{Musolesi:2007:DMM:1317425.1317433}. GPS data has also been considered for
various studies \cite{cuttone2014inferring}, \cite{5657695}. The list of
elements that have been taken into consideration for trying to understand and
predict the way in which we are conducting our daily travels is far from being
short. 

\section{Wifi based positioning}

Even from the beginning of the 21st century, research has been actively
conducted for trying to use the Wifi system in order to determine real
positioning and different databases for positioning systems have been created.
These databases usually included the positions of the Wifi access points or RF
(radio-frequency) identified fingerprints \cite{Chen:2006:PMP:2166283.2166297}
\cite{Cheng:2005:ACM:1067170.1067195} \cite{Youssef:2005:HWL:1067170.1067193}
\cite{bahl2000radar}. Modern databases for Wifi positioning are created with
information about the signal strength for the Wifi access points and can
even have information about where they were discovered.

Koo et. al. \cite{koo2011autonomous} have explored an algorithm that can help
estimate the relative positions of access points corresponding to the real
geographic configuration with the help of multidimensional scaling techniques.
Considering the fact that access points are not able to tell real distances
between themselves and other access points, the study aims to estimate the
dissimilarities between different access points using scans. They have also
conducted an experiment in an office building in order to test the proposed
algorithm and the results showed an estimation error of approximately $7$ m.

Another study conducted in this similar direction is the one by Mok et. al.
\cite{mok2007location}. The authors explore the possibility of determining the
location of a device which can scan Wifi access points based on the signal
strength that the access points are displaying at the moment of the scan. They
estimate the positioning by performing a trilateration based on the information
the device gets from multiple access points. The accuracy for their algorithm
for the conditions that were present in their experiment was of about $1-3$ m.

Athanasiou et. al. \cite{athanasiou2009utilizing} give a very clear and
concrete description for two classes of wireless positioning systems. Their work
focuses on experimenting with parameters for these algorithms in order to find
the optimal solution in terms of accuracy under realistic settings. They also
adapt a global map matching algorithm in order to extract travel time maps from
wireless data and they propose a demonstration for showing that for high
sampling frequencies, the locations identified are comparable to the ones
derived from GPS data.

The two classes of algorithms that are explored by the authors are: centroid and
fingerprinting. \textit{Centroid} is presented as the fastest method for
positioning, however it depends on having the real location of the access
points. This information is in general unavailable and as such a proposed
solution is to estimate the locations of the access points by calculating an
arithmetic mean of all the coordinates at which it was visible. The
\textit{fingerprinting} method is based on the assumption that the access points
are stable over time (they do not change positions). This leads to the fact that
at any time, a measurement at a particular location will return the same list of
access points with the same signal strengths. As such, this list can be
considered as the unique fingerprint of the location.

% TODO add 2 figures for centroid and fingerprinting
Zhang et.al. \cite{zhang2012polaris} propose an algorithm based on
fingerprinting for estimating locations that takes into consideration the fact
that the signal strength from various access points does not necessarily stay
constant throught the time. They propose a way in which a similarity between
fingerprints can be calculated in order to determine if two fingerprints are in
fact representing the same location.

These are just a selection of works that have been conducted on finding a
solution for Wifi based positioning systems. With the growth and improvement of
Wifi systems, in time all barriers can be overcome and we could have a
positioning system that is as accurate yet considerably cheaper than GPS
positioning systems.

\section{Determining the fingerprint of a location}
In order to have a better understanding about the way in which the mobile phone
users have been moving throughout the experiment, we needed to have an image of
the way a given period of time would look based on their Wifi records from
SensibleDTU. As it has been presented in Section~\ref{data_structures}, the Wifi
data we are using for the present project consists in the following fields:
user id, timestamp, SSID, BSSID, RSSI and the context. However, considering the
amount of data involved, just by looking through the log files it is almost
impossible for us to understand at what moment the user might have reached a
location and when did they leave from it. In order to be able to do this, we
have created various visualizations considering different options, different
time frames and for multiple users in order to begin to understand what the data
can tell us, what can we use, what would we need and what can we discard when
moving further to defining what makes a location.

\subsection{Signal strength over time}
\subsection{Sample density}
\subsection{Average signal strength}
\subsection{Running average signal strength}
\subsection{Signal presence}

\section{Extracting locations}
\section{Location matching}

\begin{enumerate}
  \item analazying what can determine a location fingerprint
	\begin{itemize}
		\item Plot access points’ presence over time considering their signal strength
		\item Plot number of samples of each access point over time
		\item Plot the average signal strength for various time windows for each
		access point identified for a user
		\item Plot the running average signal strength for 2,5 and 10 min time windows
		for each access point identified for a user
		\item Plot of access point presence in time bin (5 mins used for time bin)
		without considering its signal strength
	\end{itemize}
  \item identifying locations
	\begin{itemize}
		\item Networks 
		\item Hidden Markov Models 
		\item Further improvements (data cleaning)
	\end{itemize}
  \item matching locations locations
	\begin{itemize}
		\item Percentage similarity 
		\item Keeping track of previous locations 
		\item Creating fingerprints
	\end{itemize}
  \item  
\end{enumerate}
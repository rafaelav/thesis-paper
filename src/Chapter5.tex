\chapter{Locations}
\label{locations}
% 20 pages
Human mobility has been attracting a high degree of attention from numerous
study fields among which we find urban and traffic planning, traffic prediction,
the spreading of diseases and many others \cite{AsgariGB13} \cite{Brockmann08}.

The studies that have been conducted on this subject have been using various
ways to identify the travel behaviour of people. Some of them have focused on
studying the information gathered from observing the way in which money is
dispersed through time \cite{Brockmann06}, or they have been focusing in
studying the behaviour of mobile phone users by analyzing the way they move
based on the communication towers their phones are connecting to when they are
engaging in voice communication \cite{Barabasi08}. There are studies that try to
understand human mobility through the glass of social networks
\cite{yang2010using}, as it can be observed that individuals prefer to meet with
other people that are part of their community more often
\cite{Musolesi:2007:DMM:1317425.1317433}. GPS data has also been considered for
various studies \cite{cuttone2014inferring}, \cite{5657695}. The list of
elements that have been taken into consideration for trying to understand and
predict the way in which we are conducting our daily travels is far from being
short. 

\section{Wifi based positioning}

Even from the beginning of the 21st century, research has been actively
conducted for trying to use the Wifi system in order to determine real
positioning and different databases for positioning systems have been created.
These databases usually included the positions of the Wifi access points or RF
(radio-frequency) identified fingerprints \cite{Chen:2006:PMP:2166283.2166297}
\cite{Cheng:2005:ACM:1067170.1067195} \cite{Youssef:2005:HWL:1067170.1067193}
\cite{bahl2000radar}. Modern databases for Wifi positioning are created with
information about the signal strength for the Wifi access points and can
even have information about where they were discovered.

Koo et. al. \cite{koo2011autonomous} have explored an algorithm that can help
estimate the relative positions of access points corresponding to the real
geographic configuration with the help of multidimensional scaling techniques.
Considering the fact that access points are not able to tell real distances
between themselves and other access points, the study aims to estimate the
dissimilarities between different access points using scans. They have also
conducted an experiment in an office building in order to test the proposed
algorithm and the results showed an estimation error of approximately $7$ m.

Another study conducted in this similar direction is the one by Mok et. al.
\cite{mok2007location}. The authors explore the possibility of determining the
location of a device which can scan Wifi access points based on the signal
strength that the access points are displaying at the moment of the scan. They
estimate the positioning by performing a trilateration based on the information
the device gets from multiple access points. The accuracy for their algorithm
for the conditions that were present in their experiment was of about $1-3$ m.

Athanasiou et. al. \cite{athanasiou2009utilizing} give a very clear and
concrete description for two classes of wireless positioning systems. Their work
focuses on experimenting with parameters for these algorithms in order to find
the optimal solution in terms of accuracy under realistic settings. They also
adapt a global map matching algorithm in order to extract travel time maps from
wireless data and they propose a demonstration for showing that for high
sampling frequencies, the locations identified are comparable to the ones
derived from GPS data.

The two classes of algorithms that are explored by the authors are: centroid and
fingerprinting. \textit{Centroid} is presented as the fastest method for
positioning, however it depends on having the real location of the access
points. This information is in general unavailable and as such a proposed
solution is to estimate the locations of the access points by calculating an
arithmetic mean of all the coordinates at which it was visible. The
\textit{fingerprinting} method is based on the assumption that the access points
are stable over time (they do not change positions). This leads to the fact that
at any time, a measurement at a particular location will return the same list of
access points with the same signal strengths. As such, this list can be
considered as the unique fingerprint of the location.

% TODO add 2 figures for centroid and fingerprinting
Zhang et.al. \cite{zhang2012polaris} propose an algorithm based on
fingerprinting for estimating locations that takes into consideration the fact
that the signal strength from various access points does not necessarily stay
constant throught the time. They propose a way in which a similarity between
fingerprints can be calculated in order to determine if two fingerprints are in
fact representing the same location.

These are just a selection of works that have been conducted on finding a
solution for Wifi based positioning systems. With the growth and improvement of
Wifi systems, in time all barriers can be overcome and we could have a
positioning system that is as accurate yet considerably cheaper than GPS
positioning systems.

\section{Determining the fingerprint of a location}
In order to have a better understanding about the way in which the mobile phone
users have been moving throughout the experiment, we needed to have an image of
the way a given period of time would look based on their Wifi records from
SensibleDTU. As it has been presented in Section~\ref{data_structures}, the Wifi
data we are using for the present project consists in the following fields:
user id, timestamp, SSID, BSSID, RSSI and the context. However, considering the
amount of data involved, just by looking through the log files it is almost
impossible for us to understand at what moment the user might have reached a
location and when did they leave from it. In order to be able to do this, we
have created various visualizations considering different options, different
time frames and for multiple users in order to begin to understand what the data
can tell us, what can we use, what would we need and what can we discard when
moving further to defining what makes a location.

\subsection{Signal strength over time}

The first thing that we have tried to visualize was the access points (APs) that
were scanned by users' mobile phones throughout different periods of time. We
have plotted the APs and their registered signal strength for varied users in
order to see if we notice any patterns in their movements.

%userX == user 6
In Fig.~\ref{user_6_1d_lines} we can see how a day from the life of a random
user (referred to as userX) looks like. The day for which we have plotted the
data started on a Tuesday at $12:15$ pm and ends the next day right before the
same hour. The hourly intervals can be seen on the x axis, while the signal
strength values can be seen on the y axis. The legend contains the top $10$ most
popular~\footnote{An AP is more popular than another in case it appears more
times during the period of time for which the Wifi scans are analyzed}

\begin{figure}[h]
\centering
\includegraphics[height =
0.45\textwidth]{figures/user_6_sorted_1days_plot.png}
\caption{Example of the APs registered for an user throughout one day (using
connecting lines markers)}
\label{user_6_1d_lines}
\end{figure}

The steps for creating this type of visualization are as follows:
\begin{itemize}
  \item Retrieve data for the time duration for which the visualization is made
  \item Keep track of all the timestamps at which each AP has been seen and the
  AP's signal strength at that moment
  \item In case an AP is scanned no more than $2$ minutes after a it was
  previously scanned, then a line can unite the two moments in order to mark
  their proximity. If the apparitions are more than $2$ minutes apart there is
  a high possibility that there has been a location change or that the AP is
  experiencing technical problems and as such has stopped being active.
\end{itemize}

Although we have tried to visualize this type of information in various ways
(using different types of markers), we found that this way is the easiest to
interpret by people. If we leave out the lines, for example, as it can be seen
in Fig.~\ref{user_6_1d_point}, it is quite hard to interpret where location
might start or stop.

\begin{figure}[h]
\centering
\includegraphics[height =
0.45\textwidth]{figures/point_user_6_sorted_1days_plot.png}
\caption{Example of the APs registered for userX throughout one day (using
point markers)}
\label{user_6_1d_point}
\end{figure}

Other ways in which we have been experimenting with visualization for this can
be found in Appendix~\ref{appendix_signal_strength}.

By looking at Fig.~\ref{user_6_1d_lines} we are at some level able to
distinguish moments of time at which the user seems to be arriving at a
location~\footnote{For example, we can say that what we notice from Wednesday
at 10:15 until the same day at 12:15 is different than anything we can see
before that time so we can assume that it is a new location.}, however is is
hard to nice any patterns because we are only observing a single day in the life
of userX. 

% user Y = user 3
Let us look at the data gathered through $7$ days from another user's (referred
to as userY) life. The visualization for this data can be seen in
Fig.~\ref{user_3_7d}. The image gives out some very interesting information. We
can, for example, notice the repeating patterns which are dominated by the
orange, light green and blue colors. These patterns appear during the evening
and the night and we can assume that the user is spending this time at the
location which we can label ``home''.

We can notice some periods of time that are free. These free gaps like, for
example, from Monday morning until Monday evening are gaps in which no signal
was scanned and can mean that either the mobile phone was closed or that the
user decided to switch off the Wifi.

\begin{figure}[h]
\centering
\includegraphics[height =
0.45\textwidth]{figures/user_3_sorted_7days_plot.png}
\caption{Example of the APs registered for userY throughout 7 days}
\label{user_3_7d}
\end{figure}

We can also notice fragments in which the density of signals is quite high, for
example on Wednesday morning. This means that the user was located in a place
which has a large number of APs near and since we can notice a regularity in
this pattern we can assume that this place can be the University. This might
seem unlikely based on the fact that the patterns sometimes is identified during
the night, however this particular week is set in October when there are
deadlines for school projects that need to be handed in.

As we can see, these visualization can offer us a good first glance at what the
locations might be like, yet they also make us consider other things that we can
learn about the data. For example:
\begin{itemize}
  \item How many samples from each access point are received during a given time
  frame
  \item What is the average signal for various time frames for a given access
  point
  \item What are the running averages for signals from various access point
\end{itemize}
%As we can see, these visualization can offer us a good first glance at what the
%locations might be like, however, they also raise some interesting questions.
%For example:
%\begin{itemize}
%  \item Can the spikes in the signals cause any problems in determining
%  locations?
%  \item Does the sample density give us additional useful information for
%  extracting the locations?
%\end{itemize}

\subsection{Sample density}

When trying to identify locations based on the Wifi data, it is important to
only take into consideration the access points that actively contribute to the
fingerprint of the mentioned location. Before cleaning our data (as it has been
described in Section~\ref{data_cleaning}), isolated observable access points can
appear and unnecessarily burden the algorithm used for extracting the locations.
The best way to identify such access points is by analyzing the sample
density~\footnote{We define the sample density for an access point as the number
of times it appears in scans over a predefined time bin.} of the samples that
are identified during scanning.

In order to determine the sample density for each AP, we need to define a time
bin over which the sample density needs to be calculated. We have calculated the
density considering a time bin of $5$ minutes as we can assume that this amount
of time can be considered the minimum duration for which a user needs to be
situated in approximately the same place in order for us to not consider that
the location is a transition instead of a stop location.% TODO verify
% supposition

%TODO - userZ is 6 - 1 day , day 1
In Fig.~\ref{rssi_6_2nd_day} we have the different APs and their RSSI values at
the different moments when the mobile phone has identified them in the scans for
an user referred to as userZ. In Fig.~\ref{samples_6_2nd_day} we can observe the
sample density for one of the APs that are predominant during the visualized
time frame. As we can see, the number of times the AP is present in the scans
throughout the day is quite high and it is registered during numerous different
periods during the day. We can easily assume that this AP is one of the key APs
that define one of the locations the user has been associated with.

\begin{figure}[h]
\centering
\includegraphics[width =\textwidth]{figures/combinations/user_6_sorted_1days_plot_croped.png}
\caption{Example of the APs registered for userZ throughout 1 day}
\label{rssi_6_2nd_day}
\end{figure}

\begin{figure}[h]
\centering
\includegraphics[width =\textwidth]{figures/combinations/user_6_sorted_1days_plot_14280_histo.png}
\caption{Example of an AP which appears often}
\label{samples_6_2nd_day}
\end{figure}

On the opposite end as number of times it has appeared during the scans, we have
the AP in Fig.~\ref{few_samples_6_2nd_day}. As it can be seen, this AP only
appears $5$ times over a one single $5$ minute time bin. We can easily presume
that the presence or absence of this particular AP will not offer us relevant
information over the location at which the user was situated when it appeared in
the scans. This statement is also sustained by the fact that the user location
seems to be consisted from Wednesday $12:16$ up until around $13:16$ according
to what we can observe in Fig.~\ref{rssi_6_2nd_day}, even though the AP does not appear
throughout most of this time.

\begin{figure}[h]
\centering
\includegraphics[width =\textwidth]{figures/combinations/1553_modif.png}
\caption{Example of an AP that appears just a few times}
\label{few_samples_6_2nd_day}
\end{figure}

Other examples of visualizations for APs based on their sample density can be
found in Appendix~\ref{appendix_sample_density}% TODO - add stuff to appendix
% TODO - modify and add fig with only one apparitions with 5 samples in 5
% minutes -> to modify and say we take into consideration only those with more
% than 5 samples in 5 minutes

\subsection{Exploring the implications of the signal strength}
Something that is often taken into consideration during studies regarding the
determination of locations based on Wifi data is the value which indicates
the signal strength received from the various APs. The level of the signal
strength indicator can, in general, give us a good approximation of how close we
are to a particular AP. However, Wifi networks are susceptible to interferences
\cite{MahantiCWA10}, meaning that there numerous factors which can cause signals
to spike even in case the device which scans the region for AP signals does not
move. This can represent a factor of risk when including the signal strength
value in the location extraction from Wifi data as the same location could be,
at different times, be associated to an AP which has a signal strength that
oscillates based on other external factors.

In order to see if we can smooth down possible fluctuations we have employed two
mathematical tools. We have calculated the average signal strength, as well as
the running average, considering different length time bins.

\subsection{Average signal strength}

In order to calculate the average signal strength of a given AP for a given time
bin, we needed to identify all the moments of time inside the given time bin in which
the AP has been spotted during the scans. The average signal of the AP is
calculated as the sum of all the strength values that have been recorded for the
AP inside the time bin and the sum is then divided to the number of recorded
apparitions of the AP. For example, if we were to have an AP which appears 6
times inside a $5$ minutes time bin with the following RSSI values [-60, -70,
-60, -80, -90, -60], then the average signal strength for this particular time
bin for our AP would be $avg = [(-60) + (-70) + (-60) + (-80) + (-90) +
(-60)] / 6= -70$ dBm.

We have calculated the average signal for various users and various days. We
have also calculated it for different time bin length. For example, for the same
data that we can see in Fig.~\ref{rssi_6_2nd_day} and for the same AP that has
the sample density represented in Fig.~\ref{samples_6_2nd_day}, if we visualize
the non-null averages calculated for time bins of $5$ minutes, we would have the
representation in Fig.~\ref{user_6_avg_1d_5m}. The X axis records the time
while on the Y axis records the values of the averages

% TODO plot only with this AP in normal mode and maybe make some observations.
%TODO add more to appendix.

\begin{figure}[h]
\centering
\includegraphics[width =\textwidth]{figures/combinations/user_6_sorted_1days_plot_14280_avg_sig.png}
\caption{Example of average signal strength visualization for userZ}
\label{user_6_avg_1d_5m}
\end{figure}

The averages are represented by big dots symbols which appear at the beninning
of the time bin for which the average is calculated. For example, if we have
calculated an average for the interval $12:05 - 12:10$, the average is plotted
on the visualization at $12:05$. 

Additional examples of averages for different APs scanned during the same day by
userZ's mobile phone can be found in Appendix~\ref{appendix_signal_strength}.

\subsection{Running average signal strength}
\subsection{Signal presence}

\section{Extracting locations}
\section{Location matching}

\begin{enumerate}
  \item analazying what can determine a location fingerprint
	\begin{itemize}
		\item Plot the running average signal strength for 2,5 and 10 min time windows
		for each access point identified for a user
		\item Plot of access point presence in time bin (5 mins used for time bin)
		without considering its signal strength
	\end{itemize}
  \item identifying locations
	\begin{itemize}
		\item Networks 
		\item Hidden Markov Models 
		\item Further improvements (data cleaning)
	\end{itemize}
  \item matching locations locations
	\begin{itemize}
		\item Percentage similarity 
		\item Keeping track of previous locations 
		\item Creating fingerprints
	\end{itemize}
  \item  
\end{enumerate}
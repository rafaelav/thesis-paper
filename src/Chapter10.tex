\chapter{Conclusions}

This paper discusses the steps which have been taken in order to study the
inferring of human mobility patterns from Wifi data. The focus of the study has
been divided in between different areas of this subject. 

We have firstly analyzed what defines a Wifi determined location. We have taken
into consideration different approaches that can be used in order to extract
locations. We considered determining locations based on access points' BSSIDs
and RSS. This option has led to us trying to find a solution for a better noise
elimination. We have experienced with different averages that aimed to smooth
the spikes in the signal strength, however we have come to the conclusion that
for the used data a better approach is to only take into consideration the
BSSIDs of the access point and the knowledge if an access point is present or
not in a given time bin. The information extracted by analyzing when the access
points are visible has been used in order to determine stop locations.

The extraction of the locations based on the identity and presence of various
access points throughout different time frames has proved to be a challenging
and complex task. We have tested three different methods: an algorithm based on
the usage of networks, an algorithm based on $k$-means clustering and an
algorithm based on Hidden Markov Models. The first algorithm did not lead to
satisfactory results in the present study. Between the second two approaches,
the algorithm based on Hidden Markov Models has statistically had better results
for our used data, however further research is needed in order to make a
definitive assumption on which of the two algorithms can prove to be better to
use in different circumstances. Also, in order to use either of $k$-means or
Hidden Markov Models based algorithms we needed to first determine the number of
locations we were expecting the algorithms to identify. This has been achieved
by using cross validation in order to rank the results based on different
numbers of expected locations.

In order to conduct the study over a large amount of data we needed to run the
location identification algorithm on data collected during smaller time frames
(e.g. one day). The resulting locations have been compared to each other. This
was needed in order to determine if a location appeared in the results from a
different iteration. We have also compared the resulting locations extracted
from Wifi data to locations which have been extracted from analogue GPS data.
The results have been satisfactory in the sense that, even though, as expected,
GPS data can offer more detailed results about location changes, the accuracy
of the Wifi results is not to be overlooked.

By knowing the different locations and the sequence in which they occurred
throughout time we were able to calculate various entropy values for a selected
pool of users. We have also been able to determine that the users which have
been selected to be part of the present study present a high degree of
predictability as far as human mobility is concerned. This observation supports
previous results which have been made regarding this topic, yet further research
on an even bigger data set and longer period of time might be needed in order to
establish if this particular observation can be considered a fact.

We have finalized the paper by making a summary of the most important
observations that have emerged throughout our work as well as a series of
suggestions about possible future areas of interest for the present topic.

To conclude, we would like to state that the work in the present field is far
from being complete. There are many questions that still require answers and
many opportunities for improvement and we feel that the findings presented in
the current paper, as well as previous works can constitute a solid ground for
further research projects. The topic in itself is of a high interest for the
future as any results can be used to drastically improve the quality of life for
generations to come.
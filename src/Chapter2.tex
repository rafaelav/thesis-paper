\chapter{Related work}
\label{relatedwork}
There is a high interest and a huge amount of work the scientific community
dedicates to understanding the patterns of human mobility. The knowledge we can
gain from the results of this work has the potential to benefit a wide variety
of industries from the modeling and maintenance of the transportation
infrastructure, to the medical industry where we can use this knowledge in
trying to prevent the spreading of epidemics \cite{Brockmann08}.

Before starting the work on any subject we first need to understand what has
been done previously, what were the results and what have been the suggestions
for future research. The present chapter is dedicated to presenting previous
studies and the findings they offer us. The studies have been divided into $6$
sections. The first two sections focus on previous studies which provide
knowledge about mobility patterns and behaviour that can be attributed mostly to
communities and less to individuals. The following three sections present
studies that focus on individual mobility patterns. The last section presents
the results of studies that have focused on analysing and calculating the
predictability of human travel trajectories.

\section{Mobility patterns uncovered by the dissipation of bank notes}
Brockmann et. al.\cite{Brockmann06} have analyzed the human movement based on
the way bank notes were dispersed through the United States (excluding Alaska
and Hawaii). Their study shows that a relatively small percentage of bank notes
($23.6\%$) traveled for more than $800$ km, while a fraction of $19.1\%$ did not
travel for more than $50$ km even after a year of being observed. The possible
explanations the authors have given for these findings are that, in general,
people would be less inclined to leave the areas of the large cities or the
places where they usually conduct their lives.

The problem identified with this approach for tracking individuals is that the
bank notes exchange hands and the behaviour which is identified by the way they
circulate cannot be attributed to a single individual, but rather to different
ones that at some moment have had the bank note in their possession. Despite
this, the results have a high scientific value as they do identify patterns in
human travel behaviours in general.

\section{Eigenbehaviours}
Eagle et. al. analyze data of individuals and communities with the
purpose of trying to predict and cluster the daily habits and behaviour of
people \cite{Eagle09}. They consider that the behaviour of one person throughout
a day can be close to a sum of their primary eigenbahaviours throughout that day.
The results of the study have shown that when having a weighted sum calculated
for the first half of a day, the behaviour of the same person throughout the
remaining of the day can actually be approximated with $79\%$ accuracy.

The results have applicability in more fields, as they allow us to consider the
possibility of clustering people into various communities based on the
similarity of their behaviours. It goes even further, as the findings show that
this enables the possibility of calculating similarity for groups as well and
thus permitting a classification that, according to the experiment, can be
$96\%$ accurate for determining affiliations in the social network of a
particular population.

As a last observation in the paper by Eagle et. al. it is stated that
eigenbehaviours can be used in order to identify the possible friendship ties
between people. The observations in this paper have been done based on the
Reality Mining data set that tracked the behavior for $100$ individuals at MIT
for the duration of one year.

\section{Mobility patterns of mobile phone users}
Gonzalez et. al. have conducted a study \cite{Barabasi08} that deals with
observing the trajectories of over $100000$ mobile phone users with annonymized
identities. The study was conducted in order to see if there are any patterns in
our mobility habits. Among the things that have constituted subject for testing
was the return probability of individuals in a previous place. The study shows
there is, in general, a peak in the return probability after $24$, $48$ or $72$
since people have left a particular location. This shows that we tend to visit
locations periodically. This can be explained by our needs of going to places
such as work, school, grocery shops near our home etc.

The authors have also ranked the locations the mobile phone users frequented
based on the number of times they have been spotted nearby. The results for this
experiment have shown that the probability of finding someone near a location
that is ranked for them with a level $L$ can be estimated with $1/L$. Another
interesting finding that is mentioned in the paper is that, in general, people
seem to be spending the majority of their time in just a few locations, while
dividing the remaining time just between a limited number of locations that
varies for the subjects from as low as $5$ to around $50$.

The results of this study are a major indicator that individuals display a high
level of regularity and that we have a tendency to spend most of our times in
places that are familiar to us, or that require us to visit them regularly (e.g.
home, work).

\section{Human movement recorded through real traces}
Studies as the ones with the travel of bank notes or the recorded location of
mobile users through telephone towers cannot be very exact and usually cannot
reflect the real traces for the people taking part in them. They provide a very
useful estimation, however with the technology that we have access to nowadays,
we are able to record mobile phone users' real traces either through GPS or
Wifi. The data that can be acquired through these means allows us to conduct
studies that can take into consideration an even better approximation of the
real location of individuals.

In the paper by Kim et. al. \cite{Kim06}, the authors present us with a method
in which the locations of users can be estimated based on the WiFi signals that
their devices register. The experiment is conducted considering the data for a
duration of $13$ months. The user traces that have been used consist of the
trace data from Dartmouth College. The mobility traces are defined as the lists
of access points that are associated to a user's devices at a given timestamp.

The mobility traces allowed the authors to extract the tracks (locations) of the
users. They have explored three methods in which the location can be extracted
from the data. The first approach presumed the calculation of the center
(intersection of medians) of the triangle defined by the past three access
point associations of the mobile device of the user. This approach has a
downside since the devices do not necessarily change the associations in a
periodic manner. This led to the second approach which consisted in considering
a time window after which the associations needed to be updated in case new
associations have appeared during that time. The third and last approach
explored the use of Kalman filters \cite{KalmanFilter}.

In order to validate the path extractors the authors have compared the results
with GPS data. This validation has proved that the type of device used for
collecting the data can have a significant importance in how accurate the
results are as it seems that some devices can be more aggressive in updating the
associations with access points while others try to stay associated with the
same access points as long as possible before switching to new ones. This leads
to problems as different distances between users and access points considered by
different devices and as such it affects the estimated paths. The best
estimations have been given in this experiment by the approach that used the
Kalman filters, however both the other two approaches have provided fairly good
estimations as well.

Another paper which explores the travel patterns from real data is the one
written by Azevedo et. al. \cite{Azevedo09}. The authors propose another
approach for analyzing the mobility of people. They take into consideration the
following movement components: velocity, acceleration, direction angle change
and the pause time and they are using the GPS data in order to estimate the
locations of individuals. The experiment takes place in a park in Rio de Janeiro
and is done based on the data received from approximately $120$ volunteers. The
results have shown that people seem to have in general smooth trajectories without
abrupt changes.

\section{Mobility patterns in massive multiplayer online games}
Sinatra et. al. have studied the way in which users of a massive multiplayer
online game behave inside the virtual universe provided by the mentioned game
\cite{Sinatra14}. It has been established that the massive multiplayer games
provide people with a virtual reality where they can interact with others
through their characters and can, in fact, form groups and, as such, display
both individual as well as collective behaviour actions that can translate to
the non-virtual world \cite{Ball03}.

This study gives an interesting insight into the habits and actions of the
characters which are controlled by the players. Among the things the authors
have analyzed are the predictability of the characters, the entropy generated by
the mobility of the characters in the virtual universe and general strategies or
patterns that could be observed.

The game the authors have been using for the study is called
Pardus~\footnote{Further information about the game at http://www.pardus.at/}.
This game is quite complex, as it allows the manifestation of normal real-life
activities such as the creation of alliances or friendships, communication
between the players, economic related action, or even actions which have a
negative connotation such as attack of another user, removal of a friendship
link etc. The universe of the game consists in hundreds of nodes which represent
cities or sectors in the game. These virtual cities are tied to each other
through links which mark the possibility for the users to move their characters
from one place to another.

By analyzing the way in which characters have interacted through the years, the
authors have observed that the mobility of the characters through the universe
is highly predictable, as users in general will seem to be choosing a next
location in a random manner in just about $10\%$ of the cases.

\section{Entropy and predictability}
One step further from understanding the way we travel from place to place is to
predict our future locations based on a previous knowledge of our past patterns.
There has been an extensive study done in this area of the scientific playground
as well and the results which have emerged up until now are remarkable.

In the paper by Song et. al. \cite{Barabasi10}, the authors take up the
challenge of studying how predictable people can be. They analyze the mobility
patterns of mobile phone users and calculate the entropy of these users. The
locations are defined by the telephone towers the users are encountering at
hourly intervals while the trajectory of the user is given by the ordered
sequence of these towers. The real entropy of each user $i$ is calculated as
$\Sigma _{T'_{i}\subset T_{i}} P(T'_{i})log_{2}(P(T'_{i}))$, where $P(T'_{i})$
represents the probability of encountering a time-ordered subsequence $T'_{i}$
in the sequence of hourly encountered telephone towers $T_{i}$.

The results for this particular study show that, for the considered users, the
uncertainty of where they could be at a certain moment, based on the real
entropy calculated for them would be very low as they would most probably be in
one of two locations.

The authors also take a look into the maximum predictability which can be
expected for a user. Their results show that, with the right algorithm, a user's
future location can be predicted with between $80-93\%$ accuracy. This shows
that we are less spontaneous than we might think and that our mobility patterns
are, in most cases, rooted into a very well established routine.

There have been numerous other methods or experiments conducted in order to
analyze or to forecast human mobility patterns. A large number of the existing
studies focus on identifying stop locations by analyzing GPS data. There are
different types of algorithms which can be used to separate the GPS data into
stop locations, for example distance grouping \cite{cuttone2014inferring}, speed
based DBSCAN clustering \cite{Palma:2008:CAD:1363686.1363886}, the usage of GPS
signal loss \cite{Cao:2010:MSS:1920841.1920968}, the k-means clustering
\cite{Ashbrook:2003:UGL:945305.945310}, etc. Some other studies include the
Markov chain models \cite{Ross09} \cite{Liu96}, the neural networks
\cite{Liou03} or the Bayesian networks \cite{Akoush07}, finite automaton
\cite{Petzold04} or the study of human walks traces \cite{5061995}
\cite{4509740}. Most of the studies support the idea that people's actions and
travel behavior are indeed far from being random and thus the science world
needs to dedicate further effort and time in order to continue the work done for
understanding the way in which mobility patterns emerge so that we can use this
knowledge to improve our quality of life and the world we live in.
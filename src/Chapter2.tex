\chapter{Related work}
\label{relatedwork}
% 7 pages
There is a high interest and a huge amount of work the scientific community
dedicates to understanding the patterns of human mobility. The knowledge we can
gain from the results of this work has the potential to benefit a wide variety
of industries from the modeling and maintenance of the transportation
infrastructe, to the medical industry where we can use this knowledge in trying
to prevent the spreading of epidemics. \cite{Brockmann08} %TODO - add references

Various studies have been conducted in order to gain a better understanding of
the human mobility patters. These studies give us results that seem to support
each other in the idea that people are less spontaneous than they would like to
think themselves and that, indeed, our behaviour shows that we are quite rooted
into habits when it comes to the way we travel.

\section{Mobility patters uncovered by the disipation on bank notes}
Brockmann, Hufnagel and Geisel\cite{Brockmann06} have analyzed the human movement
based on the way bank notes were dispersed through the United States (excluding
Alaska and Hawaii). Their study shows that a relatively small percentage of bank
notes ($23.6\%$) traveled for more than $800$ km, while a fraction of $19.1\%$
did not traveled for more than $50$ km even after a year of being observed. The
possible explanation the authors have given for these findings are that, in
general, people would be less inclined to leave the areas of the large cities or
the places they usually conduct their lives.

The problem identified with this approach for tracking individuals is that the
bank notes exchange hands and the behaviour which is identified by the way they
circulate can't be attributed to a single individual, but rather to different
ones that at any moment have had the bank note in their possession. Despite this,
the result have a high scientific value as they do identify patterns in human
travel behaviours in general.

\section{Mobility patterns of mobile phone users}
A. L. Barabasi, M. C. Gonzalez and C. A. Hidalgo have conducted a study
\cite{Barabasi08} that deals with studying the trajectories of over $100000$
mobile phone users with anonymized identities. The study was conducted in order
to see if there are any patterns in our mobility habits. Among the things that
have been subjected to testing was the return probability of individuals in the
same place as in the past. The study shows there is, in general, a peak in the
return probability after $24$, $48$ or $72$ since they have left a particular
location. This shows that we humans tend to visit locations periodically. This
can be explaineQd by our going to places such as work, school, grocery shops
near our home etc.

The authors have also ranked the locations the mobile phone users frequented
based on the number of times they have been spotted nearby. The results for this
have shown that the probability of finding someone near a location that is
ranked for them with a level $L$ can be estimated with $1/L$. Another
interesting finding that is mentioned in the paper is that, in general, people
seem to be spending the majority of their time in just a few locations, while
diving the remaining time just between a limited number of locations that varies
for the subjects from as low as $5$ to around $50$.

There are some note worthy plots that the authors present in the paper. They can
be seen in figure{} and they show that most people travel over short distances,
yet there is a small number of people that regularly travel over big distances.

The results of this study are a major indicator that individuals display a high
level of regularity and that we have a tendency to spend most of our times in
places that are familiar to us, or that require us to visit them regularly (e.g.
home, work).

\section{Mobility patterns in massive multiplayer online games}
R. Sinatra and M. Szell have studied the way in which users of a massive
multiplayer online game behave inside the virtual universe provided by the
mentioned game \cite{Sinatra14}. It has been established that the massive
multiplayer games provide people with a virtual reality where they can interact
with others through their characters and can, in fact, form groups and, as such,
display both individual as well as collective behaviour actions that can
translate to the non-virtual world \cite{Ball03}.

This study gives an interesting insight into the habits and actions of the
characters which are controlled by the players. Among the things the authors
have analyzed are the predictability of the characters, the entropy generated by
the mobility of the characters in the virtual universe and general strategies or
patterns that could be observed.

The game the authors have been using for the study is called Pardus
\cite{Pardus}. This game is quite complex, as it allows the manifestation of
normal real-life activities such as the creation of alliances or friendships,
communication between the players, economic related action, or even actions
which have a negative connotation such as attack of another user, removal of a
friendship link etc. The universe of the game consists in hundreds of nodes
which represent cities or sectors in the game. These virtual cities are tied to
each other through links which mark the possibility for the users to move their
characters from one place to another.

By analyzing the why in which characters have interacted through the years, the
authors have observed that the mobility of the characters through the universe
is highly predictable, as users in general will seem to be choosing a random
location to visit next in just about $10\%$ of the cases.

\section{Eigenbehaviours}
N. Eagle and A. S. Pentland analyze data of individulas and communities with the
purpose of trying to predict and cluster the daily habbits and behaviour of
people \cite{Eagle09}. The consider that the behaviour of one person throughout a
day can be close to a sum of their primary eigenbahaviours throughout that day.
The results of the study have shown that when having a weighted sum calculated
for the first half of a day, the behaviour of the same person throughout the
remaining of the day can actually be approximated with $79\%$ accuracy.

The results have applicability in more fields, as they allow us to consider the
possibility of clustering people into various communities based on the
similarity of their behaviours. It goes even further, as the findings show that
this enables the possibility of calculating similarity for groups as well and
thus permitting the a classification that, according to the experiment, can be
$96\%$ accurate for determining affiliations in the social network of a
particular population.

As a last observation in the paper by N. Eagle and A. S. Pentland it is stated
that eigenbehaviours can be used in order to identify the possible friendship
ties between people. The observations in this paper have been done based on the
Reality Mining data set that tracked the behavior for 100 individuals at MIT for
the duration of one year.

\section{Human movement recorded through real traces}
Studies as the ones with the travel of bank notes or the recorded location of
mobile users through telephone is not very exact and does not reflect the real
traces for the people. They do provide a very useful estimation, however with
the technology that we have access to nowadays, we are able to record mobile
phone users' real traces either through GPS or Wifi. The data that can be
acquired through these means allows us to conduct studies that can take into
consideration a very good approximation of the real location of individuals.

In the paper by M. Kim, D. Kotz and S. Kim \cite{Kim06}, the authors present us
with a method in which the locations of users can be estimated based on the WiFi
signals that their devices register. The experiment is conducted cosndiering the
data for a duration of $13$ months. The user traces that have been used consist
of the trace data from the Darmouth College. The mobility traces are defined as
the lists of access points that are associated to a user's devices at a given
timestamp.

The mobility traces allowed the authors to extract the tracks (locations) of the
users. They have explored three methods in which the location can be extracted
from the data. The first approach presumed the calculation of the center
(intersection of medians) of the triangled defined by the past three access
point associations of the mobile device of the user. This approach has a
downside since the devices do not necessarily change the associations in a
periodic manner. This lead to the second approach which consisted in considering
a time window after which the associations needed to be updated in case new
associations have appeared during that time. The third and last approach
explored the use of Kalman filters \cite{KalmanFilter}.

The validation the path extractors the authors have compared the results with
GPS data. This validation has prove that the type of the used device has at
the moment a significant importance in how accurate the results can be as it seems
that some devices can be more aggressive in updating the associations with
access points while others try to stay associated with the same access points as
long as possible before switching to new ones. This leads to problems as
different distances between users and access points considered by different
devices and as such it affects the estimated paths. The best estimations have
been given in this experiment by the approach that used the Kalman filters,
however both the other two approaches have provided fairly good estimations as
well.

Another paper which explores the travel patterns from real data is the one
written by T. S. Azevedo, R. L. Bezerra, C. A. V. Campos and L. F. M. de Moraes
\cite{Azevedo09}. The authors propose another approach for analyzing the mobility
of people. They take into consideration the following movement components:
velocity, acceleration, direction angle change and the pause time and they are
using the GPS data in order to estimate the locations of individuals. The
experiment takes place in a park in Rio de Janeiro and is done based on the data
received from around $120$ volunteers. The results have shown that people seem
to have in general smooth trajectories without abrupt changes.
% TODO - social network theory TODO - radio landscape obs - Perez Penichet paper
% TODO - Levi flight and people vs animals

% TODO - locations and identifying locations paragraph TODO - 2-3 papers on
% identifying locations TODO - paragraph about predictability and entropies
% (Lu13 - approaching the limits of predictability) TODO - 2 papers on this
\section{Entropy and predictability}
One step further from understanding the way we travel from place to place is to
predict our future locations based on a previous knowledge our our past
patterns. There has been an extensive study done in this area of the scientific
playground as well and the results which have emerged up until now are
remarcable.

In the paper by C. Song, Z. Qu, N. Blumm and A. L. Barabasi \cite{Barabasi10},
the authors take up the challenge of studying how predictable people can be.
They analyze the mobility patterns of mobile phone users and calculate the
entropy of these users. The locations are defined by the telephone towers the
users are encountering at hourly intervals and the trajectory of the user is
given by the ordered sequence of these towers. The real entropy of each user i
is calculated as $\Sigma _{T'_{i}\subset T_{i}} P(T'_{i})log_{2}(P(T'_{i}))$,
where $P(T'_{i})$ represents the probability of encountering a time-ordered
subsequence $T'_{i}$ in the sequence of hourly encountered telephone towers
$T_{i}$.

The results for this particular study show that, for the considered users, the
uncertainty of where they could be at a certain moment, based on the real
entropy calculated for them would be very low as they would most probably be in
one of two locations.
% The results show that, for most of the users that were part of this particular
% study, the entropy S has been calculated to be around $0.8$. This means that %
% the real uncertainty when it comes to estimating where a user can be at a
% certain moment is $2^{0.8}$ which means $1.74$, so less than $2$ locations.

The authors also take a look into the maximum predictability which can be
expected for a user. Their results show that, with the right algorithm, a user's
future location can be predicted with between $80-93\%$ accuracy. This shows
that we are less spontaneous than we might think and that our mobility patterns
are, in most cases, rooted into a very well established routine.

There have been numerous other methods or experiments conducted in order to
analyze or to forecast human mobility patterns. Some of these methods include the
Markov chain models \cite{Ross09} \cite{Liu96}, the neural networks \cite{Liou03}
or the Bayesian networks \cite{Akoush07} as well as some that work with finite
automaton \cite{Petzold04}. Most of the studies support the idea that people's
actions and travel behavior is indeed far from being random and thus the science
world needs to dedicate further effort and time in order to use this knowledge
in order to improve our quality of life and the world we live in.

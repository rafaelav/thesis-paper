\chapter{Related work}

There is a high interest and a huge amount of work the scientific community
dedicates to understanding the patterns of human mobility. The knowledge we
can gain from the results of this work has the potential to benefit a wide
variety of industries from the modeling and maintenance of the transportation infrastructe,
to the medical industry where we can use this knowledge in trying to prevent the
spreading of epidemies. %TODO - add references

Various studies have been conducted in order to gain a better understanding of
the human mobility patters. These studies give us results that seem to support
each other in the idea that people are less spontaneous than they would like to
think themselves and that, indeed, our behaviour shows that we are quite rooted
into habbits when it comes to the way we travel.

Brockmann, Hufnagel and Geisel\ref{Brockmann06} have
analyzed the human movement based on the way bank notes were dispersed through
the United States (excluding Alaska and Hawaii). Their study shows that a
relatively small percentage of bank notes ($23.6\%$) traveled for more than
$800$ km, while a fraction of $19.1\%$ did not traveled for more than $50$ km
even after a year of being observed. The possible explanation the authors have
given for these findins are that, in general, people would be less inclined to
leave the areas of the large cities or the places they usually conduct their
lives.

% TODO - calls and distances
% TODO - online behaviour
% TODO - eigenbehaviour
% TODO - social network theory
% TODO - radio landscape obs - Perez Penichet paper
%TODO - Levi flight and people vs animals

% TODO - locations and identifying locations paragraph
% TODO - 2-3 papers on identifying locations
% TODO - paragraph about predictability and entropies
% TODO - 2 papers on this


\chapter{Related work}
% 7 pages
There is a high interest and a huge amount of work the scientific community
dedicates to understanding the patterns of human mobility. The knowledge we
can gain from the results of this work has the potential to benefit a wide
variety of industries from the modeling and maintenance of the transportation infrastructe,
to the medical industry where we can use this knowledge in trying to prevent the
spreading of epidemies. \ref{Brockmann08} %TODO - add references

Various studies have been conducted in order to gain a better understanding of
the human mobility patters. These studies give us results that seem to support
each other in the idea that people are less spontaneous than they would like to
think themselves and that, indeed, our behaviour shows that we are quite rooted
into habbits when it comes to the way we travel.

\section{Mobility patters uncovered by the disipation on bank notes}
Brockmann, Hufnagel and Geisel\ref{Brockmann06} have
analyzed the human movement based on the way bank notes were dispersed through
the United States (excluding Alaska and Hawaii). Their study shows that a
relatively small percentage of bank notes ($23.6\%$) traveled for more than
$800$ km, while a fraction of $19.1\%$ did not traveled for more than $50$ km
even after a year of being observed. The possible explanation the authors have
given for these findins are that, in general, people would be less inclined to
leave the areas of the large cities or the places they usually conduct their
lives.

\section{Mobility patterns of mobile phone users}
A. L. Barabasi, M. C. Gonzalez and C. A. Hidalgo have conducted a study
\ref{Barabasi08} that deals with studying the trajectories of over $100000$
mobile phone users with anonymized identities. The study was conducted in order
to see if there are any patterns in our mobility habbits. Among the things that
have been subjected to testing was the return probability of individuals in the
same place as in the past. The study shows there is, in general, a peak in the
return probability after $24$, $48$ or $72$ since they have left a particular
location. This shows that we humans tend to visit locations periodically. This
can be explaineQd by our going to places such as work, school, grocery shops
near our home etc.

The authors have also ranked the locations the mobile phone users frequented
based on the number of times they have been spotted nearby. The results for this
have shown that the probability of finding someone near a location that is
ranked for them with a level $L$ can be estimated with $1/L$. Another
interesting finding that is mentioned in the paper is that, in general, people
seem to be spending the majority of their time in just a few locations, while
diving the remaining time just between a limited number of locations that varies
for the subjects from as low as $5$ to around $50$.

There are some note worthy plots that the authors present in the paper. They can
be seen in figure{} and they show that most people travel over short distances,
yet there is a small number of people that regularly travel over big distances.
%This is something that might raise the interesting question of whether those
%people who travel over longer distances, are less predictable than those who do
%not. - something to referenciate when writting about the other paper with
% limits of predictability

The results of this study are a major indicator that individuals display a high
level of regularity and that we have a tendency to spend most of our times in
places that are familiar to us, or that require us to visit them regularly
(e.g. home, work).

\section{Mobility patterns in massive multiplayer online games}
R. Sinatra and M. Szell have studied the way in which users of a massive
multiplayer online game behave inside the virtuale universe provided by the
mentioned game \ref{Sinatra14}. It has been established that the massive
multiplayer games provide people with a virtual reality where they can interact
with others through their characters and can, in fact, form groups and, as
such, display both individual as well as collective behaviour actions that can
translate to the non-virtual world \ref{Ball03}.

This study gives an interesting insight into the habbits and actions of the
characters which are controlled by the players. Among the things the authors
have analyzed are the predictability of the characters, the entropy generated
by the mobility of the characters in the virtual universe and general
strategies or patterns that could be observed.

The game the authors have been using for the study is called
Pardus~\footnote{http://en.wikipedia.org/wiki/Pardus\_(browser\_game)}. This
game is quite complexe, as it allows the manifestation of normal real-life
activities such as the creation of alliances or friendships, communication
between the players, economic related action, or even actions which have a
negative conotation such as attack of another user, removal of a friendship
link etc. The universe of the game consists in hundrets of nodes which represent
cities or sectors in the game. These virtual cities are tied to each other
through links which mark the posibility for the users to move their characters
from one place to another.

By analyzing the why in which characters have interacted through the years, the
authors have observed that the mobility of the characters through the univers is
highly predictable, as users in general will seem to be chosing a random
location to visit next in just about $10\%$ of the cases.

% TODO - eigenbehaviour
% TODO - social network theory
% TODO - radio landscape obs - Perez Penichet paper
%TODO - Levi flight and people vs animals

Other methods that have been used in trying to forcast human mobility patterns
include Markov chain models \ref{Ross09} \ref{Liu96}, neural networks
\ref{Liou03}, Bayesian networks \ref{Akoush07}, and finite automaton
\ref{Petzold04}.

% TODO - locations and identifying locations paragraph
% TODO - 2-3 papers on identifying locations
% TODO - paragraph about predictability and entropies (Lu13 - approaching the
% limits of predictability) TODO - 2 papers on this


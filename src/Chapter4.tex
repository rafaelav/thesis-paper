\chapter{Data analysis and clean up}
% 3 pages TODO - keep writing
The present project uses data that has been selected from the database of the
SensibleDTU experiment. The data is fully anonymized and the users that have
been a part of the study have been chosen randomly from the database.

\section{Data statistics}
% TODO(update the number of users if needed)
We use the data collected from $131$ users from the SensibleDTU database. The
students that have been selected for the present study had data collected for a
period of almost a year.~\footnote{The starting time of collection for the 2012
deployment of SernsibleDTU is October $1^{st}$ 2012 and the end is September
$1^{st}$ 2013.}

The application that is installed on the smartphones of the students who are
part of the experiment is configured to scan periodically (around every $15$
seconds) for Wifi networks, however, it is also set to record the scans which
are triggered by any of the other applications that are present on the mobile
phone.

\section{Wifi and GPS data}
For the present study we are not using all the fields that are accessible from
the database of collected information. The study's aim is to analyze the
predictability and patterns in the human mobility and as such we need
information that can tell us the locations of the users that are part of the
study. For this we are accessing information about the Wifi associations for the
selected group of users. The results reagrding the users' locations over time
are afterwards compared with recorded GPS locations and as such we are accessing
this information from the database as well.

% For the Wifi data, we are retriving from the database the fields in
% Fig.~\figure{}.

The user (first) field gives us information about what user we are currently
observing. The real identities of the users are concealed and replaced by an ID
which is unique for each of them.

The timestamp (second) field gives us information about the moment of time at
which the scan occured and for which the information is gathered. The time
format is Unix timestamp.~\footnote{The Unix time stamp represents a way in
which time can be tracked as the total number of seconds starting from January
$1^{st}$, 1970 at UTC and a particular date and time.} This timestamp can be
easily manipulated and converted to any other timestamp format in Python by
using the datetime module that can be found in the Python Standard Library
\cite{PSL}.

The ssid (third) field stands for Service Set Identifier and it represents the
unique ID that can be used in order to identify the wireless networks. This
identifier is responsible for the correct sending of data when multiple wireless
networks overlap.

The bssid (forth) field stands for Basic Service Set identifier and it
represents the MAC address of a wireless access point.

The rssi (fifth) field stands for Received Signal Strength Indication and it
represents the strength for a signal picked up by the mobile phone from an
access point. The rssi values in our case are registered as the real signal
strength recorded in dBm and are therefore negative values. As such, the signal
is stronger when the value recorded for it is closer to $0$.

The context (sixth) field is based in the ssid and it translates to the
possibilities presented in %Fig.~\figure{}

% TODO - add data stuff and description for GPS filds
\section{Noise elimination}

\begin{enumerate}
\item why working with wifi data;
\item data structure and fields;
\item how much data was analized
\item how was the data prepared (noise elimination)
\end{enumerate}
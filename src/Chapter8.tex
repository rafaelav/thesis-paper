\chapter{An evaluation of Wifi positioning accuracy} 

The evolution of technology during the present days allows us to make use of
devices that register and process high amounts of contextual information about
their owners. Among these devices we can find our smartphones. Smartphones have
become a commodity without which we can hardly imagine our day to day life. We
use them for connecting and communicating with our friends, we use them for
recreational activities such as ``surfing'' the web, or in order to keep
ourselves up to date with the news in various domains that have captured our
interest. However, aside from allowing us to access all these types of
information while connecting us with the world around us, our smartphones have
another ability which not so many of us completely understand. They can collect
a large amount of contextual information about us. A strong example of how such
information can be used in the interest of the mobile phone's owner is provided
by the Google Now \cite{GN} service. This service uses various contextual
information that we make available through our Android \cite{ANDR} phones in
order to provide us with important information like traffic statistics for when
we are expected to travel, weather information for the destination we should be
arriving at, or news about our favorite television show, for example.

An example of contextual information that can be retrieved from our smartphones
is our locations, or our traveling patterns. This is possible due to the GPS
positioning that our smartphones allow. There are numerous studies which focus
on the possibility of extracting human mobility patterns from the GPS data
provided by mobile phones; among these studies we find
\cite{Montoliu:2010:DHP:1899475.1899487},
\cite{cuttone2014inferring},
\cite{Zhou:2007:DPM:1247715.1247718}, \cite{Ashbrook:2003:UGL:945305.945310}.

The database from SensibleDTU (Section~\ref{sensible_dtu}) contains, as it has
been mentioned previously, numerous types of information among which we can also
find GPS coordinates of the users that have accepted to be part of the
scientific project. For the present study, we had the opportunity to employ part
of this data in order to compare the results we obtain about the user locations
from their Wifi data with possible stop locations provided from their GPS data.

\section{Extracting stop locations from GPS data}

\section{Comparing results}
% 5 pags
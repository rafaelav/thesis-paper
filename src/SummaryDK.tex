\chapter{Summary - Danish}

I forskningsverdenen er forståelsen af menneskers bevægelsesmønstre et varmt
emne, der har været omdrejningspunktet for talrige studier. Kilden til
datamateriale for disse analyser spænder bredt, fra GPS data, data indsamlet via
telekommunikationsmaster og sågar informationer indsamlet i forbindelse med
distributionen af pengesedler. I denne afhandling vil vi se nærmere på, hvordan
bevægelsesmønstre kan udledes fra WiFi data. Vi analyserer metoder, som via WiFi
data kan bruges til at identificere stop-punkter, vi sammenholder disse
resultater med stop-punkter identificeret ved hjælp af GPS data og vi undersøger
i hvor høj grad menneskers bevægelsesmønstre kan forudsiges.
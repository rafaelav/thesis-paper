\chapter{Summary - Danish}

Forståelsen af menneskers bevægelsesmønstre er et aktivt forskningsfelt og
omdrejningspunkt for talrige studier. Datamaterialet bag disse analyser spænder
bredt, fra GPS-data til data indsamlet via telekommunikationsmaster, og helt til
bevægelsesmønstre udledt på baggrund af diffusion af pengesedler. I denne
afhandling ser vi nærmere på hvordan bevægelsesmønstre kan udledes fra Wifi
data. Vi analyserer metoder, som på baggrund Wifi data, kan bruges til at
identificere "stop locations" -- steder hvor en person opholder sig i længere
tid -- og sammenholder disse resultater med "stop locations" identificeret ved
hjælp af GPS data. Endelig undersøger vi i hvilken grad menneskers
bevægelsesmønstre kan forudsiges.

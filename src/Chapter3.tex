\chapter{Prerequisites and tools}
% 2 pag
In order to research the way in which people travel we firstly need to have
access to a database of information that can be used for this purpose. As it was
mentioned in Chapter \ref{relatedwork}, scientists have been trying in numerous
way to identify and work with location information. During our study, we have
dedicated our time in working with information about the access points that were
visible to the users' mobile phones throughout their day. This has allowed use
to implement and analyze different ways in which locations can be extracted from
such information.

\section{SensibleDTU}
\label{sensible_dtu}

The data we are using is part of a large-scale study that aims to make
observations based on the lives of volunteering students - the Copenhagen
Network Study \cite{Stopczynski14m}. The main aim of this project is to offer an
extensible framework for different studies that can help us have a better
understanding of the human nature and how personal data can be used to analyse
individual behaviour in order to promote self-awareness and positive behaviour
changes. The deployments for this study from 2012 and 2013 are based at the
Technical University of Denmark and are named SensibleDTU. 

A factor that represents a definite strength of this project is that it allows
the collection of data phase to co-exist with various analysis phases. The
platform allows the conducting of controlled studies which can be distributed to
participants with the help of the smartphone software. Due to the way in which
the platform is designed, the participants can be divided in different groups
and these groups can be exposed to different stimuli in order to allow the
analysis of results gathered for a multitude of possible experiments. Another
benefit of being able to work with the provided data as soon as it is received
is that data can be monitored without unnecessary delay. By doing this, the
quality of the date can be evaluated for both the level of an individual user as
well as for the whole dataset. The results of a qualitative evaluation allow the
researches to understand if the collected data in its form is sufficient for
answering the various research questions. The real time processing of the data
has the benefit that it allows the researches to create applications and
services designed for the participants. These services can be used for receiving
feedback and thus improve the understanding of what can be improved in order to
maximize the usefulness of how the data is used in the ongoing studies, but they
can also be used for offering the participants the chance to access the results
of their own data and thus helping them understand themselves better.

The students that consented to being volunteers for SensibleDTU have received
smartphones that are able to track different aspects of their lives and through
which they can interact with the system. The big number of
volunteers~\footnote{Up to the moment at which the present paper has been
written, there have been deployed over 1000 smartphones to students who wanted
to take part in the study.} has allowed the gathering of a considerable amount
of data regarding the mobile phone users' behaviour. 

The data is collected from a variety of sources. Participants in the various
studies can receive questionnaires which can be focused on getting information
about their socioeconomic background, habits or even psychological
traces~\footnote{For example, a survey was presented to the participants in 2012
consisting of over 90 questions. In 2013 an addition of over 300 questions were
ask per participant. The questioned targeted different aspects from working
habits and various socio-econimic factors to Big Five Inventory measuring
personality traits \cite{John99} and self-esteem.}. Sensor data is collected
throughout the days at regular intervals as well as Wifi data. SensibleDTU
offers the participants the option to authorize the collection of data from
Facebook. This type of data can be used on order to create friendship graphs and
to follow the interactions between various participants in the studies.
Qualitative data is collected in order to gather feedback from the participants
as well as understanding what can keep their interest in the project at a high
level.

Since the majority of the collected information about the students is sensitive
\cite{Stopczynski14p}, keeping the data secure is and has been a top priority
from the beginning of the experiment. The data is annonymzed and stored securely
and the students that are part of the experiment have access to tools that allow
them to see what data they are sharing, what it is done with this data and that
allow them to control how much they want to share. The combination between the
measurements which are taken in order to keep the data anonymized as well as the
fact that the users have control and understanding of their own data ensures a
good security level.


\section{Using Wifi data for defining locations}

The main aim of the present project is to explore the human mobility and the
predictability of our traveling destinations based on the Wifi data gathered for
a selection of participants from the SensibleDTU study.  

The market of smart hand held devices is constantly and rapidly growing. This
has opened a new market for applications which take the location of the user
into consideration for offering him or her personalized services \cite{alsehly2010improving}.
In order for information about locations to be offered to such applications, the need for
a system that can identify these locations at a low cost and in a sufficiently
accurate matter has developed. 

``Wireless Positioning Systems (WPS) provide a position estimate based on the
radio signals received at a given location (measurement), and a known radio map
of the environment.''\cite{athanasiou2009utilizing} The possibility of
determining locations based on Wifi data is based on the fact that each access
point that can be found inside a network has an unique id attributed to it
(BSSID) and that each location receives a different signal strength from an
access point given the distance in between them. Thus, the received signal
strength (RSS) can also prove to be an important factor in the definition of a
location based on Wifi data and various algorithm for determining position based
on Wifi have already been designed and used \cite{athanasiou2009utilizing}.

Studies show that Wifi positioning can be acceptably good, as it can have an
accuracy of 1-4 m in an indoor environment and between 10-40 m for an outdoor
environment, depending on the number of access points which define the area as
well as the presence of interferences that can affect the accuracy for the
positioning \cite{Cheng:2005:ACM:1067170.1067195} \cite{mok2007location}.
Considering these numbers, the increase in the usage of smartphones and
the cost wise advantages presented by Wifi positioning systems, ensure that the
studies on human mobility include and should keep on including the data that can
be acquired from Wifi networks.

\section{Implementation tools}
Before starting the work on the present research, we have carefully taken into
consideration possible tools that can be useful in our work.

The scripts that are used for analyzing, transforming and working with the data
are developed in Python. The reasons behind using Python instead of any other
programming language are numerous. Python is elegant and simple to use, it
allows fast development and the code can be easily adapted and reused. Due to
its high scalability, it is the perfect choice for both large and small
projects, being easily extensible at the same time. Another very important
reason for using Python is that there is a large number of libraries that can be
used with it and that allow the visualization or handling of big
data.~\footnote{Examples of libraries and packages used: numpy, pickle,
datetime, sympy etc.}

The matplotlib library \cite{Mplib} is a very useful tool when dealing with
visualizations which can be made directly from Python scripts that are used for
data analysis. It can be used in order to generate plots, histograms,
scatterplots, bar charts and many additional types of figures in an easy way.

An additional tool that has been used for the present project is Gephi
\cite{Gephi}. Gephi is a platform that allows the exploration and handling of
various networks and graphs. Further information on how this tool has proved
helpful can be found in Chapter \ref{locations}.

Scikit-learn \cite{SL} provides ready, simple to use and effective tools that
can be used for data mining and analysis. The provided tools include regression
algorithms, implementations for various models (e.g. Hidden Markov Models,
further details in \ref{hmm_section}), pre-processing for various data,
clustering algorithms (e.g. k-means, further details in \ref{k-means}) and many
other equally useful mechanisms and implementations.

Pandas \cite{Pandas} is a powerful and easy-to-use Python library that allows
the work with data structures through offering various data analysis tools.
Among the tools provided by this library we can find the intelligent data
alignment, tools for reading and writing data of various formats, merging and
joining of data sets, grouping and reshaping of data and many others.
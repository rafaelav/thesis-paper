\chapter{Prerequisits and tools}
% 2 pag
In order to research the way in which people travel we firstly need to have
access to a database of information that can be used for this purpose. As it was
mentioned in Chapter \ref{relatedwork}, scientists have been trying in numerous
way to identify and work with location information. During our study, we have
dedicated our time in working with information about the access points that were
visible to the users' mobile phones throught their day. This has allowed use to
implement and analyze different ways in which locations can be extarcted from
such information.

\section{SensibleDTU}
The data we are using is part of a large-scale study that aims to make
observations based on the lives of volunteering students - the Copenhagen
Network Study. The data is colected from a variety of sources. Some of them
require the volunteers to interact with the system through questionnaires and
others track them automatically through their smartphones. The aim of this
project is to offer an extensible framework for different studies. The
deployments from 2012 and 2013 are based at the Technical University of Denmark
and are named SensibleDTU \cite{Stopczynski14m}.

The students that consented to being volunteers for this ambitious project have
received smartphones that are able to track different aspects of their lives and
through which they can interact with the system. The big number of
volunteers~\footnote{During the second iteration, there have been deployed
aproximately 1000 smartphones to students who wanted to take part in the study.}
has allowed the gathering of a considerable amount of data regarding the mobile
phone users' behaviour.

The data gathered for the SensibleDTU experiment consists in data gathered
through questionnaires~\footnote{A survey was presented to the participants in
2012 consisting of over 90 questions. In 2013 an addition of over 300 questions
were ask per participant. The questioned targeted different aspects from working
habbits and various socio-econimic factors to Big Five Inventory measuring
personality traits \cite{John99} and self-esteem.}, Facebook
data~\footnote{Participants have the option of allowing the gathering of
Facebook data such as friendships and various interactions such as likes,
statuses etc.}, sensor data, qualitative data and Wifi data.

Since the majority of the collected information about the students is sensitive
\cite{Stopczynski14p}, keeping the data secure is and has been a top priority
from the beginning of the experiment. The data is annonymzed and stored securely
and the students that are part of the experiment have access to tools that allow
them to see what data are they sharing, what it is done with this data and that
allow them to control how much they want to share.

\section{Wifi fingerprinting and location estimation}
% TODO - why considering this method
% searchi polaris in mail for additional references for fingerprinting
% (like 4 of them)

\section{Implementation tools}
Before starting the work on the present research, we have overviewd possible
tools that can be useful in our work.

The scripts that are used for analyzing, transforming and working with the data
are developed in Python. The reasons behing using Python instead of any other
programming language are numerous. Python is elegant and simple to use, it
allows fast development and the code can be easily adapted and reused. Due to
its high scalability, it is the perfect choice for both large and small
projects, being easily extensible at the same time. Another very important
reason for using Python is that there is a large number of libraries that can be
used with it and that allow the visualization or handling of big
data.~\footnote{Examples of libraries and packages used: numpy, matplotlib,
pickle, datetime, sympy etc.}

An additional tool that has been used for the present project is Gephi
\cite{Gephi}. Gephi is a platform that allows the exploration and handeling of
various networks and graphs. Further information on how this tool has proven
helpfull can be found in Chapter \ref{locations}.
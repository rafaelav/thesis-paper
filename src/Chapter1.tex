\chapter{Introduction}

The United Nations (UN) Department of Economic and Social Affairs' Population
Division is in charge of preparing once every two years an estimation of what we
are to expect the growth rate for the world population to be in the following
years. Occasionally, world population projections over a longer period of time
are created as well. In $2004$, the UN has made predictions about the trends in
population growth up until $2300$ \cite{UNWp}. The predictions show that the
population will increase to reach a peak of $9.22$ billion by the time of $2075$
and then it will decrease slowly to $8.97$ billion by $2300$. These estimations
bring up different aspects regarding our quality of life which is a topic that
will only increase in importance with the population growth. For example, we
need to think about more efficient ways in which we can develop the urban
regions or our transportation infrastructure in order to ensure that space is
used in a responsible and optimal manner. Also, it is important to understand
how epidemics spread and how they can be contained in order to avoid devastating
pandemics from tacking place.

The rapid evolution of technology has equipped us with tools that can be used in
order to gather large amounts of information about human behaviour and mobility
patterns. This information can be employed by scientists in research and study
programs in order to obtain facts and results that can be used to solve possible
problems which are sure to appear due to the rise in population that seems to be
facing us. Current studies about the predictability of human mobility uncovered
remarkable findings which suggest that we might be less spontaneous when it
comes to choosing our destinations than we might think we are. These results can
prove to be of tremendous help in the making of decisions about transportation
infrastructure and they can even give us an insight on how diseases are
spreading from region to region.

Due to the importance of the results that can emerge from studying the
predictability of human mobility, we have conducted a study that is focused on
this topic, mainly the inferring of mobility patterns from Wifi data. In order
to understand what defines a location we analyze different methods that can be
used for extracting stop locations from the gathered Wifi data. We discuss and
propose a solution for determining if two identified locations represent in fact
the same geographical stop location. This helps us construct a long term image
of the travel trajectories associated to the users who are providing the data.
We compare the results we have, our stop locations obtained from the Wifi data,
to results generated using GPS data and we explore the degree of predictability
of human travel trajectories based on the image we are able to create about the
mobility patterns over a longer period of time.

The detailed explanation of all steps made during our work, as well as the
results and observations we have come across are structured into eight different
chapters. Chapter $2$ (Related work) presents previous findings and studies
conducted on the topic. Chapter $3$ (Prerequisites and tools) presents the
elements that have contributed to making this work possible: data gathered with
the help of volunteers, implementation, visualization and data analysis tools
etc. Chapter $4$ (Data processing) presents the data that has been used during
our work as well as the way in which we have eliminated interferences and noise
from the received data. Chapter $5$ (Extracting locations from Wifi data)
presents all the steps that have been taken from the data analysis up to the
analysis of the different algorithms that we have experimented with in order to
extract the locations from the Wifi data. Chapter $6$ (Location matching)
presents the techniques we have considered in order to estimate if two given
locations that have been discovered by using a location extraction algorithm can
be catalogued as being the same location based on defining characteristics.
Chapter $7$ (Entropy and predictability) presents the calculations made in order
to determine the different entropy values and predictability that can be
attributed to the users based on the location information that emerged from
their data as well as what these values represent. Chapter $8$ (An evaluation
of Wifi positioning accuracy) presents the comparison made between the results
we have obtained from the Wifi data and the stop locations which can be
extracted from the analogue GPS data. Chapter $9$ (Discussion and future work)
presents a summary of the results and observations which have emerged during
our work as well as proposed topics and directions for further work on the
present subject.

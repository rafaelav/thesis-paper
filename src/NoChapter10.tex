\chapter{Future work}

The work that has been detailed in the present paper is concerned with analyzing
the mobility patterns and the predictability of human travel trajectories. The
proposed approaches and their results as well as references to previous work
prove that this subject is and need to remain of a high interest for the science
world. Because of this future work is necessary in order to gain further
knowledge about it.

The work we have conducted in order to extract locations from Wifi data has
focused on three different types of algorithms
(Section~\ref{extracting_location}). Among them, there is an algorithm that is
based on the construction of networks from Wifi information, networks that can
be used in order to estimate what characteristics define a given location and as
such enable us to identify the locations. During our study, we did not reach
acceptable results by the use of this method, however, further research is
needed in order to determine if the idea behind this approach can produce note
worthy results. The benefits are that, by identifying clear characteristics of
the locations, this enables the construction of an algorithm that can have a
lower execution time for extracting locations from large amounts of data.

The results we have obtained for predictability of human mobility based on the
data from the selected group of volunteers are in agreement with results
presented in previous studies. The data we have used for this part of the study
has been collected through a period of one month and is provided by a focus
group of $65$ people who have in common the fact that they are students at the
Technical University of Denmark. Further research on a larger group of people
possibly with various backgrounds could give additional interesting results and
would be needed in order to make defined affirmations about emerging mobility
and predictability patterns.

We have taken into consideration various possibilities that can be used in order
to determine if two locations which have been identified in different
iterations by a location extraction algorithm can, in fact, be considered to be
the same location. Further work can be done in order to optimize the solution
which has been chosen for the present study. Even though the estimations which it
gives are accurate most of the times, additional improvements can be made and
further research can lead to the development of a new method that can exceed the
benefits of the one that has been used in this case.

During our work we have debated what is a good approximation for the time an
user can be expected to stay at a given location in order to consider it a stop
location rather than a transit location. We have proposed the use of a $5$
minute time frame, however further studies can be conducted in order to estimate
what is the average time a person spends at a given location based on the
location type. This can lead to further considerations and adaptations for
algorithms that try to extract stop locations both from Wifi data, but also from
other types of data, for example GPS data.

Additional work can be done for improving our understanding of inferences and
noise that can cause problems when working with Wifi data. The present paper
presented a few techniques which have been used in order to eliminate possible
noise, however, future improvements can be added for obtaining data which has an
even higher qualitative value.

Another interesting focus point for future research can be represented by a
method of estimating the expected number of locations for a given time frame.
The present paper has presented the use of cross validation as a method of
determining the best possible estimation, however, Hidden Markov Models
constitute a highly effective model which, possibly, with further research can
be able to eliminate the use of estimation and allow the exact determination of
the number of locations which can be observed during a given sequence of data.
Also other methods can be explored and tested for achieving the best results.

Clearly the path of exploring the subject of human mobility and predictability
of human travel trajectories is far from reaching an end. Numerous studies, the
present one included, have been conducted on this topic, however there is still
a lot of work that can and needs to be done in order to improve the estimations
and to gain new knowledge on the subject, as questions still remain unanswered.
The topic continues to remain of high interest as the results have a
applicability potential that expands over a variety of fields and domains.
